\chapter{Some Not-so-crazy Math}

\section{Total and Partial Derivatives}
To understand the different kinds of derivatives, let's say we have a function $\rho(t, x(t), p(t))$ which, in general, depends on the location $x(t)$ and momentum $p(t)$ plus the time $t.$ A key observation is that the location $x(t)$ and momentum $p(t)$ are functions of $t$ too. Therefore, we need to be extremely careful what we mean when we calculate the derivative with respect to the time $t .$
$$
\frac{d \rho}{d t}=\lim _{\Delta t \rightarrow 0} \frac{\rho(t+\Delta t, x(t+\Delta t), p(t+\Delta t))-\rho(t, x(t), p(t))}{\Delta t}
$$
\textbf{The result is the total rate of change of $\rho$}.
$$
\frac{\partial \rho}{\partial t}=\lim _{\Delta t \rightarrow 0} \frac{\rho(t+\Delta t, x(t), p(t))-\rho(t, x(t), p(t))}{\Delta t}
$$
The key difference is that we only vary $t$ if it appears explicitly
in $\rho$ but not if it only appears implicitly because $x(t)$ and $p(t)$
also depend on $t .$ Thus
$$
\frac{d \rho}{d t}=\frac{\partial \rho}{\partial x} \frac{d x}{d t}+\frac{\partial \rho}{\partial p} \frac{d p}{d t}+\frac{\partial \rho}{\partial t}
$$
\section{Taylor Expansion}
In general, we want to estimate the value of some function $f(x)$ at some value of $x$ by using our knowledge of the function's value at some fixed point $a .$ The Taylor series then reads
\begin{equation}
\begin{aligned}
f(x)=& \sum_{n=0}^{\infty} \frac{f^{(n)}(a)(x-a)^{n}}{n !} \\
=& \frac{f^{(0)}(a)(x-a)^{0}}{0 !}+\frac{f^{(1)}(a)(x-a)^{1}}{1 !}+\frac{f^{(2)}(a)(x-a)^{2}}{2 !} \\
&+\frac{f^{(3)}(a)(x-a)^{3}}{3 !}+\ldots
\end{aligned}
\end{equation}
or
\begin{equation}
f(x+a)=f(x)+(a \cdot \partial) f(x)+\frac{1}{2}(a \cdot \partial)^{2} f(x)+\cdots
\end{equation}
Taylor expansion of a scalar field (function $f$ that maps $\mathbb{R}^{n}$ to $\mathbb{R}$).Now, identify $\partial f / \partial t$ as $\hat{\boldsymbol{n}} \cdot \nabla f .$ In addition, see that $t \hat{\boldsymbol{n}}=\boldsymbol{x}-\boldsymbol{x}_{0} .$ Some clever recombining of terms gives
\begin{equation}
f(x)=f\left(x_{0}\right)+\left.\left(x-x_{0}\right) \cdot \nabla f\right|_{x_{0}}+\left.\frac{1}{2}\left(\left[x-x_{0}\right] \cdot \nabla\right)^{2} f\right|_{x_{0}}+\ldots
\end{equation}
and
\begin{equation}
\hat{\boldsymbol{n}} \cdot \nabla=\partial_{t}
\end{equation}

\section{Vector Identities}
\begin{equation}
\begin{aligned}
&\vec{\nabla} \cdot(\vec{\nabla} \times \vec{A}) \equiv \operatorname{div}(\operatorname{rot} \vec{A})=(\vec{\nabla} \times \vec{\nabla}) \cdot \vec{A} \equiv 0\\
&\vec{\nabla} \times(\vec{\nabla} \varphi) \equiv \operatorname{rot} \operatorname{grad} \varphi=(\vec{\nabla} \times \vec{\nabla}) \varphi \equiv 0
\end{aligned}
\end{equation}
\begin{equation}
\begin{aligned}
&\vec{\nabla} \cdot(\vec{A} \varphi)=\varphi \vec{\nabla} \cdot \vec{A}+\vec{A} \cdot \vec{\nabla} \varphi \quad \Longleftrightarrow \quad \operatorname{div}(\vec{A} \varphi)=\varphi \operatorname{div} \vec{A}+\vec{A} \cdot \operatorname{grad} \varphi\\
&\vec{\nabla} \times(\vec{A} \varphi)=\varphi \vec{\nabla} \times \vec{A}-\vec{A} \times \vec{\nabla} \varphi \quad \Longleftrightarrow \quad \operatorname{rot}(\vec{A} \varphi)=\varphi \operatorname{rot} \vec{A}-\vec{A} \times \operatorname{grad} \varphi\\
&\vec{\nabla} \cdot(\vec{A} \times \vec{B})=\vec{B} \cdot(\vec{\nabla} \times \vec{A})-\vec{A} \cdot(\vec{\nabla} \times \vec{B}) \quad \Longleftrightarrow \quad \operatorname{div}(\vec{A} \times \vec{B})=\vec{B} \cdot \operatorname{rot} \vec{A}-\vec{A} \cdot \operatorname{rot} \vec{B}
\end{aligned}
\end{equation}
\begin{equation}
\begin{aligned}
\vec{\nabla} \times(\vec{A} \times \vec{B}) &=(\vec{B} \cdot \vec{\nabla}) \vec{A}-(\vec{A} \cdot \vec{\nabla}) \vec{B}+\vec{A}(\vec{\nabla} \cdot \vec{B})-\vec{B}(\vec{\nabla} \cdot \vec{A}) \\
& \Longleftrightarrow \operatorname{rot}(\vec{A} \times \vec{B})=(\vec{B} \operatorname{grad}) \vec{A}-(\vec{A} \operatorname{grad}) \vec{B}+\vec{A}(\operatorname{div} \vec{B})-\vec{B}(\operatorname{div} \vec{A})
\end{aligned}
\end{equation}
\begin{equation}
\begin{aligned}
\vec{\nabla}(\vec{A} \cdot \vec{B})=&(\vec{B} \cdot \vec{\nabla}) \vec{A}+(\vec{A} \cdot \vec{\nabla}) \vec{B}+\vec{A} \times(\vec{\nabla} \times \vec{B})+\vec{B} \times(\vec{\nabla} \times \vec{A}) \\
& \Longleftrightarrow \operatorname{grad}(\vec{A} \cdot \vec{B})=(\vec{B} \cdot \operatorname{grad}) \vec{A}+(\vec{A} \cdot \operatorname{grad}) \vec{B}+\vec{A} \times \operatorname{rot} \vec{B}+\vec{B} \times \operatorname{rot} \vec{A}
\end{aligned}
\end{equation}
\begin{equation}
\begin{aligned}
&\vec{\nabla} \cdot(\vec{\nabla} \varphi) \equiv \operatorname{div}(\operatorname{grad} \varphi) \equiv \Delta \varphi=\frac{\partial^{2} \varphi}{\partial x^{2}}+\frac{\partial^{2} \varphi}{\partial y^{2}}+\frac{\partial^{2} \varphi}{\partial z^{2}}, \quad \Delta=\text { Laplace Operator }\\
&\vec{\nabla} \times(\vec{\nabla} \times \vec{A}) \equiv \operatorname{rot}(\operatorname{rot} \vec{A})=\vec{\nabla}(\vec{\nabla} \cdot \vec{A})-(\vec{\nabla} \cdot \vec{\nabla}) \vec{A} \equiv \operatorname{grad} \operatorname{div} \vec{A}-\Delta \vec{A}
\end{aligned}
\end{equation}