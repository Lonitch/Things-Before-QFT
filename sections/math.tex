\chapter{Some Not-so-crazy Math}

\section{Total and Partial Derivatives}
To understand the different kinds of derivatives, let's say we have a function $\rho(t, x(t), p(t))$ which, in general, depends on the location $x(t)$ and momentum $p(t)$ plus the time $t.$ A key observation is that the location $x(t)$ and momentum $p(t)$ are functions of $t$ too. Therefore, we need to be extremely careful what we mean when we calculate the derivative with respect to the time $t .$
$$
\frac{d \rho}{d t}=\lim _{\Delta t \rightarrow 0} \frac{\rho(t+\Delta t, x(t+\Delta t), p(t+\Delta t))-\rho(t, x(t), p(t))}{\Delta t}
$$
\textbf{The result is the total rate of change of $\rho$}.
$$
\frac{\partial \rho}{\partial t}=\lim _{\Delta t \rightarrow 0} \frac{\rho(t+\Delta t, x(t), p(t))-\rho(t, x(t), p(t))}{\Delta t}
$$
The key difference is that we only vary $t$ if it appears explicitly
in $\rho$ but not if it only appears implicitly because $x(t)$ and $p(t)$
also depend on $t .$ Thus
$$
\frac{d \rho}{d t}=\frac{\partial \rho}{\partial x} \frac{d x}{d t}+\frac{\partial \rho}{\partial p} \frac{d p}{d t}+\frac{\partial \rho}{\partial t}
$$
\section{Taylor Expansion}
In general, we want to estimate the value of some function $f(x)$ at some value of $x$ by using our knowledge of the function's value at some fixed point $a .$ The Taylor series then reads
\begin{equation}
\begin{aligned}
f(x)=& \sum_{n=0}^{\infty} \frac{f^{(n)}(a)(x-a)^{n}}{n !} \\
=& \frac{f^{(0)}(a)(x-a)^{0}}{0 !}+\frac{f^{(1)}(a)(x-a)^{1}}{1 !}+\frac{f^{(2)}(a)(x-a)^{2}}{2 !} \\
&+\frac{f^{(3)}(a)(x-a)^{3}}{3 !}+\ldots
\end{aligned}
\end{equation}
or
\begin{equation}
f(x+a)=f(x)+(a \cdot \partial) f(x)+\frac{1}{2}(a \cdot \partial)^{2} f(x)+\cdots
\end{equation}
Taylor expansion of a scalar field (function $f$ that maps $\mathbb{R}^{n}$ to $\mathbb{R}$).Now, identify $\partial f / \partial t$ as $\hat{\boldsymbol{n}} \cdot \nabla f .$ In addition, see that $t \hat{\boldsymbol{n}}=\boldsymbol{x}-\boldsymbol{x}_{0} .$ Some clever recombining of terms gives
\begin{equation}
f(x)=f\left(x_{0}\right)+\left.\left(x-x_{0}\right) \cdot \nabla f\right|_{x_{0}}+\left.\frac{1}{2}\left(\left[x-x_{0}\right] \cdot \nabla\right)^{2} f\right|_{x_{0}}+\ldots
\end{equation}
and
\begin{equation}
\hat{\boldsymbol{n}} \cdot \nabla=\partial_{t}
\end{equation}