\chapter{Quantum Mechaaaaanics}
\textit{This is a cheat sheet for QM. The focus here is not giving comprehensive explanations but summarizing concise "quantum" bullet points.}

\section{State vectors and Their Conjugates}
1. \textbf{Quantum operators}: quantities we can measure.

2. A state vector represents an eigenstate of a particular operator \textbf{if measuring the associated quantity always yields the same result}:

3. operator $\times$ eigenstate $=$ eigenvalue $\times$ eigenstate

4. \textbf{state vector} is usually a linear combination of eigenstates:
$$
|\Psi\rangle=\sum_{i} a_{i}\left|o_{i}\right\rangle= a_{1}\left|o_{1}\right\rangle+ a_{2}\left|o_{2}\right\rangle+\dots
$$
and probability to measure $o_{i}=\left|a_{i}\right|^{2}$.

5. A \textbf{conjugated state vectors ("bra")} is a Hermitian conjugated ket:
$$
\langle\Psi|=| \Psi\rangle^{\dagger}=\left(|\Psi\rangle^{\star}\right)^{T}
$$
6. The \textbf{expectation value of an observable is then}
\begin{equation}
\text { expectation value }=\langle\Psi|\hat{O}| \Psi\rangle
\label{expectation}
\end{equation}
7. \textbf{Wave function:} function $\psi(x)$ that we get by expanding a state vector in terms of position eigenstates:
\begin{equation}
|\Psi\rangle=\int d x \psi(x)|x\rangle
\end{equation}
8. \textbf{Schrödinger equation:}
\begin{equation}
i \hbar \partial_{t}|\Psi\rangle=-\frac{\hbar^{2} \partial_{i}^{2}}{2 m}|\Psi\rangle+ V(\hat{x})|\Psi\rangle
\end{equation}
9. The frequency of the wave associated with a given particle is directly related to its energy as:
\begin{equation}
v=\frac{E}{h}
\end{equation}
10. the wavelength is directly related to its momentum 
\begin{equation}
\lambda=\frac{h}{p}
\end{equation}

\section{Quantum Framework}
\subsection{Wave Function}
Use \textbf{location projector operators}, we formulate the \textbf{wave function} as:
\begin{qt}
    \begin{equation}
\begin{aligned}
|\Psi\rangle &=\int d x|x\rangle\langle x | \Psi\rangle \\
& \equiv \int d x \Psi(x)|x\rangle
\end{aligned}
\end{equation}
where $\Psi(x) \equiv\langle x | \Psi\rangle$
\end{qt}
\bluep{The physical interpretation of $\Psi(x)$ is again as a probability amplitude.} 

Using momentum basis states, we obtain the \textbf{momentum representation} of the wave function:
\begin{equation}
\begin{aligned}
|\Psi\rangle &=\int d p|p\rangle\langle p | \Psi\rangle \\
& \equiv \int d p \Psi(p)|p\rangle
\end{aligned}
\end{equation}
The operators $\int d x|x\rangle\langle x|$, $\int d p|p\rangle\langle p|$ are the \textbf{identity operator}, i.e., operators that do not change anything.

An important idea is that by using a given wave function $\Psi(x)$ we can immediately calculate important quantities like the expectation value: 
\begin{qt}
    $$
\begin{aligned}
\langle\Psi|\hat{O}| \Psi\rangle &=\left\langle\Psi\left|\hat{O} \int d x \Psi(x)\right| x\right\rangle \\
&=\int d x^{\prime}\left\langle x^{\prime}\left|\Psi^{+}\left(x^{\prime}\right) O \int d x \Psi(x)\right| x\right\rangle \\
&=\int d x^{\prime} \int d x\left\langle x^{\prime}\left|\Psi^{\dagger}\left(x^{\prime}\right) \hat{O} \Psi(x)\right| x\right\rangle|x\rangle
\end{aligned}
$$
$$
\begin{aligned}
&=\int d x^{\prime} \int d x \Psi^{\dagger}\left(x^{\prime}\right) O \Psi(x) \underbrace{\left\langle x^{\prime} | x\right\rangle}_{=\delta\left(x-x^{\prime}\right)}\\
&=\int d x \Psi^{+}(x) O \Psi(x)
\end{aligned}
$$
and
$$
\begin{aligned}
\langle\Phi | \Psi\rangle &=\int d x^{\prime}\left\langle x^{\prime}\left|\Phi\left(x^{\prime}\right) \int d x \Psi(x)\right| x\right\rangle \\
&=\int d x^{\prime} \int d x\left\langle x^{\prime}\left|\Phi^{\dagger}\left(x^{\prime}\right) \Psi(x)\right| x\right\rangle \\
&=\int d x^{\prime} \int d x \Phi^{\dagger}\left(x^{\prime}\right) \Psi(x) \underbrace{\left\langle x^{\prime} | x\right\rangle}_{=\delta\left(x-x^{\prime}\right)} \\
&=\int d x \Phi^{\dagger}(x) \Psi(x)
\end{aligned}
$$
\end{qt}
\subsection{Quantum Operators}
We now talk more about the operators from a perspective of \textbf{symmetries}. The part of mathematics which deals with symmetries is called \textbf{group theory}. A group is a set of transformations which fulfill special rules plus an operation that tells us how to combine the transformations. Also, \textbf{we only need one special part of group theory, namely the part that deals with \redp{continuous symmetries}.}
\begin{qt}
    There is one property that makes continuous symmetries especially
nice to deal with:
\begin{center}
    \textbf{they have elements which are arbitrarily close to the identity transformation.}
\end{center}
\end{qt}
For example, think about the symmetries of a circle. Any rotation about the origin is a symmetry of a circle. Therefore, a rotation extremely close to the identity transformation, say a rotation by $0.000001^{\circ},$ is a symmetry of the circle.

Mathematically, we write an element $g$ close to the identity $I$ as:
\begin{equation}
g(\epsilon)=I+\epsilon G
\end{equation}
where $\epsilon$ is a really, really small number and $G$ is an object, called a \redp{generator}. In the smallest possible case, such transformations are called \textbf{infinitesimal transformations}.

Let’s return to our discussion about rotations. Many small rotations in one direction are equivalent to one big rotation in the same direction. Mathematically, we can write \bluep{the idea of repeating a small transformation many times} as follows:
\begin{equation}
h(\theta)=(I+\epsilon G)(I+\epsilon G)(I+\epsilon G) \ldots=(I+\epsilon G)^{k}
\end{equation}
If $\theta$ denotes some finite transformation parameter, e.q.,$50^o$ or so, and $N$ is some huge number that makes sure we are close to the identity. We write $g$ as:
$$
g(\theta)=I+\frac{\theta}{N} G
$$
At the limit of $N\rightarrow\infty$, we have 
$$
h(\theta)=\lim _{N \rightarrow \infty}\left(I+\frac{\theta}{N} G\right)^{N}=e^{\theta G}
$$
\begin{qt}
    The bottom line is that the object $G$ generates the finite transformation $h .$ This is why we call objects like this \textbf{generators}.
\end{qt}
Let's consider a function $f(x, t)$ and assume that our goal is to generate a spatial translation such that $T f(x, t)=f(x+a, t)$ The following generator do the job:
\begin{qt}
    \begin{equation}
G_{\mathrm{xtrans}}=\partial_{x}
\end{equation}
and
\begin{equation}
\begin{aligned}
e^{a G_{\mathrm{xtrans}}} f(x, t) &=\left(1+a G_{\mathrm{xtrans}}+\frac{a^{2}}{2} G_{\mathrm{xtrans}}^{2}+\ldots\right) f(x, t) \\
&=\left(1+a \partial_{x}+\frac{a^{2}}{2} \partial_{x}^{2}+\ldots\right) f(x, t) \\
&=f(x+a, t)
\end{aligned}
\end{equation}
\textbf{$G_{\mathrm{xtrans}}=\partial_{x}$ generates spatial translations.}
\end{qt}
Analogously, \redp{\textbf{$G_{\text {ttrans }}=\partial_{t}$ generates temporal translation}}.
\begin{qt}
    \begin{center}
        The core of each continuous symmetry is the corresponding generator.
        
        \textit{\textbf{quantum operator $\leftrightarrow$ generator of symmetry}}
    \end{center}
\end{qt}
\begin{qt}
\begin{center}
    1. momentum $\hat{p}_{i} \leftrightarrow$ g'nerator of spatial translations $\left(-i \hbar \partial_{i}\right)$
    
    2. energy $\hat{E} \leftrightarrow$ generator of temporal translations $\left(i \hbar \partial_{t}\right)$
    
    3. Since there is no symmetry connected to the conservation of position, the position operator stays as $\hat{x}$
\end{center}
\end{qt}
\begin{qt}
\textbf{Canonical commutation relation} reads:
\begin{equation}
\left[\hat{p}_{i}, \hat{x}_{j}\right]=-i \hbar \delta_{i j}
\end{equation}
\end{qt}
In physical terms, we cannot measure the position and the momentum of a particle at the same time with arbitrary precision. We can now also understand how this comes about in our framework. We identified the momentum
operator as the generator of spatial translations.\bluep{Thus, each time we measure the momentum, we perform a tiny
spatial translation.}