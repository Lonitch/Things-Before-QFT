\chapter{Special Relativity}
\section{Maxwell Equations and Lorentz Transformations}
Our purpose in this section is to use tensor methods to clarify the relation between electrodynamics and relativity. Let's first write Lorentz law and Maxwell equations in the Gaussian system of units, where electric and magnetic fields have the same dimension. The quantity $c$ is the speed of light:
\begin{qt}
    \begin{equation}
\mathbf{F}=q \mathbf{E}+\frac{q}{c} \mathbf{v} \times \mathbf{H}
\end{equation}
\begin{equation}
\begin{aligned}
\operatorname{div} \mathbf{E} &=4 \pi \rho, & & \operatorname{rot} \mathbf{H}=\frac{1}{c} \frac{\partial \mathbf{E}}{\partial t}+\frac{4 \pi}{c} \mathbf{j} \\
\operatorname{div} \mathbf{H} &=0, & \operatorname{rot} \mathbf{E} &=-\frac{1}{c} \frac{\partial \mathbf{H}}{\partial t}
\end{aligned}
\end{equation}
where $\mathbf{j}=\mathbf{v} \cdot \rho(t, \mathbf{r})$ is the current density.
\end{qt}
\redp{The equations above contradict the laws of Newtonian mechanics.} To see this, consider a pair of point-like charges $q_1$,$q_2$ that interact by means of Coulomb force. Now if the same two charges in another inertial frame that moves with the constant speed $-\mathbf{v}$ in the direction orthogonal to the line between the two charges.

According to the standard logic of classical mechanics, the electric Coulomb force should not change because of the choice of inertial frame. At the same time, in the new frame, each of the charges will move with a constant speed $+\mathbf{v}$. But now the two charges \bluep{become currents and hence each of them creates a magnetic field.} This magnetic field will act on another charge according to Lorentz's law; therefore, there will be an increase in the force acting on each of the charges. On the other hand, according to the Galilean principle, the force cannot change when we switch from one inertial frame to another such frame—hence
this example shows that there is a contradiction. Thus, \textbf{this sytem of equations is not invariant under the Galilean transformations between inertial references frames}:
\begin{equation}
\mathbf{r}^{\prime}=\mathbf{r}+\mathbf{v} t, \quad \mathbf{t}^{\prime}=\mathbf{t}
\label{Galilean-trans}
\end{equation}
The successful solution the conflict between mechanics and electromagnetism lies in the generalization of the transformation(\ref{Galilean-trans}). \bluep{The new transformation should be performed in such a way that the tranformed laws are compatible with the Maxwell equations.} \textbf{\redp{Being compatible here means that changing from one inertial frame to another one does not change the form of these equations, while the quantities such as $\mathbf{H}, \mathbf{E}, \rho, \mathbf{j},$ and $\mathbf{v}$ do transform in a controllable tensorical way.}}

Consider Maxwell equations without sources:
\begin{equation}
\begin{aligned}
\operatorname{div} \mathbf{E} &=0, & & \operatorname{rot} \mathbf{H}=\frac{1}{c} \frac{\partial \mathbf{E}}{\partial t} \\
\operatorname{div} \mathbf{H} &=0, & \operatorname{rot} \mathbf{E} &=-\frac{1}{c} \frac{\partial \mathbf{H}}{\partial t}
\end{aligned}
\label{Maxwell-noSource}
\end{equation}
In order to obtain the wave equation, we remember the formula:
$$
\operatorname{rot}(\operatorname{rot} \mathbf{E})=\operatorname{grad}(\operatorname{div} \mathbf{E})-\Delta \mathbf{E}
$$
Apply the $rot$ operator to both side of the second equation in (\ref{Maxwell-noSource}). Using the fact of $div\mathbf{E}=0$, we arrive at the equation:
\begin{equation}
-\Delta \mathbf{E}=-\frac{1}{c} \operatorname{rot} \frac{\partial \mathbf{H}}{\partial t}=-\frac{1}{c} \frac{\partial \operatorname{rot} \mathbf{H}}{\partial t}=-\frac{1}{c^{2}} \frac{\partial^{2} \mathbf{E}}{\partial t^{2}}
\end{equation}
or
\begin{equation}
\square \mathbf{E}=0, \quad \square=\frac{1}{c} \frac{\partial^{2}}{\partial t^{2}}-\Delta
\end{equation}
where we introduce the d'Alembert operator $\square$. Similarly, we have $\square \mathbf{H}=0$.

\textbf{Another important observation is that} the lines of the magnetic field $\mathbf{H}$ are always closed and do not have their ends on some sources. This feature can be provided by assuming that there is a \textbf{vector potential field $\mathbf{A}$} such that
\begin{equation}
\mathbf{H}=\operatorname{rot} \mathbf{A}
\end{equation}
Contrary to this, in the presence of sources the lines of electric field $\mathbf{E}$ always have their ends on the point-like sources or at the infinity. This feature corresponds to the definition
\begin{equation}
\mathbf{E}=-\frac{1}{c} \frac{\partial \mathbf{A}}{\partial t}-\operatorname{grad} \varphi
\end{equation}
where $\varphi$ is an additional scalar potential.
\subsection{Invariant Interval and Minkowski Space}
Let us start by defining the Minkowski space, which has the coordinates
\begin{equation}
x^{0}=c t, \quad x^{1}=x, \quad x^{2}=y, \quad x^{3}=z
\end{equation}
The standard abbreviations for Minkowski space are $M_{1,3}$ or $M_4$. In what follows we shall use Greek indices for the four-dimensional notations in Minkowski space, e.g., $\alpha, \beta, \ldots, \mu$
$0,1,2,3$ and Latin indices for the space variables, e.g., $a, b, \ldots, i j, \cdots=1,2,3$ One can write, for instance, $x^{\mu}=\left(x^{0}, x^{i}\right)$ or $x^{\mu}=\left(x^{0}, \mathbf{r}\right)$.

The point in Minkowski space is frequently called the \textbf{event}. The event is defined by time and space coordinates. Correspondingly, the interval between two infinitesimally separated events has the form
\begin{equation}
d s^{2}=\left(d x^{0}\right)^{2}-\left(d x^{1}\right)^{2}-\left(d x^{2}\right)^{2}-\left(d x^{3}\right)^{2}=c^{2} d t^{2}-d l^{2}
\end{equation}
where $d l^{2}=d x^{2}+d y^{2}+d z^{2}$ is the space interval between the two close events. The last formula can be also written in the form
\begin{equation}
d s^{2}=\eta_{\mu v} d x^{\mu} d x^{v}, \quad \text { where } \quad \eta_{\mu v}=\operatorname{diag}(1,-1,-1,-1)
\end{equation}
is the metric of the $p$seudo-Euclidean Minkowski space, which is also called space-
time. 

\redp{The nonlinear changes of coordinates that involve both space and time variables can be consistently considered in general relativity, but better avoided in special relativity.} The reason is that this type of the change in the space–time coordinates means that we go to the non-inertial reference frame. Under the change of inertial frame, we can use the definition of $ds^2$ above arrive at the \textbf{invariance of the interval}:
\begin{qt}
    \begin{equation}
d s^{\prime 2}=d s^{2}
\label{ds-invariance}
\end{equation}
\end{qt}
There is a clear classification of the intervals between the two events into three distinct groups: The events of the first type have a positive square of the interval, $s^2_{12}>0$, and are called \textbf{time-like}. Themotion from one event to another one in this case can be done with the constant velocity $v<c$. In particular, the two events that occur at the same space point are separated by a time-like interval. The second type of separation is the \textbf{light-like interval}, $s^2_{12}=0$. The motion with constant velocity between the two intervals can be performed only with the speed of light $v=c$. The third type of separation is by the \textbf{space-like interval}, $s^2_{12}=0$. This case only happens when $v>c$.
\subsection{Lorentz Transformations}
Consider the two inertial reference frames, the static one $K$ and another $K^{\prime}$, which are moving with the constant velocity $\mathbf{v}$ with respect to $K$. Since the space is isotropic, without loss of generality it is possible to assume that the direction of the velocity is along the axis X.

Now we resolve Eq.(\ref{ds-invariance}) in the two-dimensional case:
$$
c^{2} d t^{\prime 2}-d x^{2}=c^{2} d t^{2}-d x^{2}
$$
The solution can be easily found by analogy with the Euclidean rotations. We know that the continuous transformation that preserves the length. One can make an analytic continuation of time $ct=ix^4$ and remember that 
$$
\sin i x=-\frac{\sinh x}{i}=(-1) \cdot \frac{\sinh x}{i}=\left(i^{2}\right) \cdot \frac{\sinh x}{i}=i \sinh x
$$
$$
\cos (i z)=\cosh z
$$
we have the hyperbolic rotation:
\begin{equation}
\left(\begin{array}{c}
{c t^{\prime}} \\
{x^{\prime}}
\end{array}\right)=\left(\begin{array}{c}
{\cosh \psi \sinh \psi} \\
{\sinh \psi \cosh \psi}
\end{array}\right)\left(\begin{array}{l}
{c t} \\
{x}
\end{array}\right)
\end{equation}
Since $dx=0$, we have
$$
v=c \beta=\frac{d x^{\prime}}{d t^{\prime}}=-c \tanh \psi
$$
Thus,
$$
\cosh \psi=\frac{1}{\sqrt{1-\beta^{2}}}, \quad \sinh \psi=-\frac{\beta}{\sqrt{1-\beta^{2}}}, \quad \text { where } \quad \beta=\frac{v}{c}
$$
Now we are in a position to write the Lorentz transformations in terms of $v$ or $\beta$,
\begin{qt}
    \begin{equation}
c t^{\prime}=\frac{c t-\beta x}{\sqrt{1-\beta^{2}}}, \quad x^{\prime}=\frac{x-\beta c t}{\sqrt{1-\beta^{2}}}, \quad y^{\prime}=y, \quad z^{\prime}=z
\label{lorentz-trans}
\end{equation}
\end{qt}
The inverse transformation is:
\begin{qt}
    \begin{equation}
c t=\frac{c t^{\prime}+\beta x^{\prime}}{\sqrt{1-\beta^{2}}}, \quad x=\frac{x^{\prime}+\beta c t^{\prime}}{\sqrt{1-\beta^{2}}}, \quad y=y^{\prime}, \quad z=z^{\prime}
\label{inverse-lorentz}
\end{equation}
\end{qt}

\section{Laws of Relativistic Mechanics}
The Lorentz transformations (also called boosts) and orthogonal space transformations (rotations and inversion of coordinates) \textbf{do not change the form of the metric.} An important relation concerning $ds$ for a massive particle that moves with the speed $v=\beta c\leq c$ is as follows:
\begin{equation}
d s=\sqrt{c^{2} d t^{2}-d l^{2}}=\sqrt{c^{2} d t^{2}-v^{2} d t^{2}}=c \sqrt{1-\beta^{2}} d t
\end{equation}
Along with the interval, one can introduce a \textbf{\redp{proper time}} of the particle, which is defined as $d \tau=d s / c=\sqrt{1-\beta^{2}} d t$.

The evolution of a particle in the Minkowski space $M_{4}$ is described by \redp{the world line $x^{\mu}(\lambda)$ where $\lambda$ is a parameter along the curve}. The tangent vector along the curve is defined as $k^{\mu}=d x^{\mu} / d \lambda .$ One can choose $\lambda$ in different ways, for example, take $d \lambda=d x^{0} .$ However, in many cases, it is most useful to choose $d \lambda=d s .$ Indeed, this choice works well only for the time-like curves, where the tangent vector $k^{\mu}$ to the world line of a particle is time-like, $k^{\mu} k_{\mu} \geq 0 .$ Certainly, \textbf{the interval is not the best choice in the case of a light-like tangent vector}, because in this case $d s^{2}=0 .$ One of the possibilities for this case is to use \bluep{the time variable, $t$ or $x^{0} .$}

Consider the world line of a massive particle, which (as we shall see in what follows) has a time-like tangent vector. Taking $d \lambda=d s,$ we arrive at the \redp{four-velocity of the particle},
\begin{qt}
    \begin{equation}
u^{\mu}=\frac{d x^{\mu}}{d s}
\end{equation}
and
\begin{equation}
u_{\mu} u^{\mu}=\eta_{\mu \nu} u^{\mu} u^{\nu}=\frac{d s^{2}}{d s^{2}} \equiv 1
\end{equation}
\end{qt}
It is interesting to find the relation between four-velocity and usual 3D velocity of
the particle, $\mathbf{v}=d\mathbf{r}/dt$:
\begin{equation}
u^{i}=\frac{d x^{i}}{d s}=\frac{d x^{i}}{c d \tau}=\frac{1}{c \sqrt{1-\beta^{2}}} \frac{d x^{i}}{d t}=\frac{v^{i}}{c \sqrt{1-\beta^{2}}}
\label{relativistic-ui}
\end{equation}
and 
\begin{equation}
u^{0}=\frac{d x^{0}}{d s}=\frac{1}{\sqrt{1-\beta^{2}}}
\label{relativistic-u0}
\end{equation}
The next relevant quantity is \redp{four-acceleration,}
\begin{qt}
    \begin{equation}
\omega^{\mu}=\frac{d u^{\mu}}{d s}=\frac{d^{2} x^{\mu}}{d s^{2}}
\end{equation}
\end{qt}
Since $u_{\mu} u^{\mu}=\eta_{\mu \nu} u^{\mu} u^{\nu}=\frac{d s^{2}}{d s^{2}} \equiv 1$, we have
$$
\frac{d(u_{\mu} u^{\mu})}{ds}=2\eta_{\mu\nu}u^{\mu}\frac{du^{\nu}}{ds}=0
$$
Thus,
\begin{qt}
    \begin{equation}
u_{\mu} \omega^{\mu} \equiv 0
\end{equation}
\end{qt}
The last thing about kinematics is to obtain the formula for \textbf{\redp{summing velocities}}.Let us suppose that the inertial reference frame $K^{\prime}$ moves with velocity $\mathbf{V}$ with respect to the static inertial frame $K .$ Consider the particle that moves with the constant velocity $\mathbf{v}^{\prime}$ with respect to $K^{\prime} .$ What is its velocity with respect to $K ?$ In classical mechanics, we have
$$
\mathbf{v}=\mathbf{V}+\mathbf{v}^{\prime}
$$
In the frame $K'$, we have the relations:
\begin{equation}
    d x^{\prime}=v_{x}^{\prime} d t^{\prime}, \quad d y^{\prime}=v_{y}^{\prime} d t^{\prime}, \quad d z^{\prime}=v_{z}^{\prime} d t^{\prime}
    \label{k-k'}
\end{equation}
Without loss of generality one can assume that the velocity $V$ has only $X$ component, $\mathbf{V}=V \hat{\mathbf{i}} .$ Then, according to the differential version of (\ref{inverse-lorentz}),
\begin{equation}
    c d t=\frac{c d t^{\prime}+\beta d x^{\prime}}{\sqrt{1-\beta^{2}}}, \quad d x=\frac{d x^{\prime}+\beta c d t^{\prime}}{\sqrt{1-\beta^{2}}}, \quad d y=d y^{\prime}, \quad d z=d z^{\prime}
    \label{k-k'2}
\end{equation}
where $\beta=V/c$. Combining (\ref{k-k'}) and (\ref{k-k'2}), we arrive at the result:
\begin{qt}
    \begin{equation}
v_{x}=\frac{v_{x}^{\prime}+V}{1+V v_{x}^{\prime} / c^{2}}, \quad v_{y}=\frac{\sqrt{1-\beta^{2}} v_{y}^{\prime}}{1+V v_{x}^{\prime} / c^{2}}, \quad v_{z}=\frac{\sqrt{1-\beta^{2}} v_{z}^{\prime}}{1+V v_{x}^{\prime} / c^{2}}
\end{equation}
\end{qt}
\begin{example}
 Consider the pure space transformations (e.g., rotations and parity transformation, that is, the simultaneous inversion of all space coordinates) of $F_{\mu v}=-F_{v \mu} .$ Show that with respect to these particular transformations, the components of $F_{\mu\nu}$ transform as the components of usual $V_i$ and axial $A_i$ vectors, such that
 \begin{equation}
F_{\mu v}=\left(\begin{array}{cccc}
{0} & {V_{1}} & {V_{2}} & {V_{3}} \\
{-V_{1}} & {0} & {A_{3}} & {-A_{2}} \\
{-V_{2}}&{-A_{3}} & {0} & {A_{1}} \\
{-V_{3}} & {A_{2}} & {-A_{1}} & {0}
\end{array}\right)
\end{equation}
or 
\begin{equation}
 F_{0 i}=V_{i} \text { and } F_{i j}=\epsilon_{i j k} A_{k}
\end{equation}
\end{example}
\textbf{Solution:}
Consider the transformation of the desired type, with $x^{0}=x^{0} .$ Then
$$
F_{0 i}^{\prime}\left(x^{\prime}\right)=\frac{\partial x^{\alpha}}{\partial x^{0}} \frac{\partial x^{\beta}}{\partial x^{i^{\prime}}} F_{\alpha \beta}(x)
$$
The first factor is nonzero only when $\alpha=0,$ and the second only when $\beta=j$ therefore
$$
F_{0 i}^{\prime}=\frac{\partial x^{j}}{\partial x^{i^{\prime}}} F_{0 j}
$$
The last formula means that the components $F_{0 i}$ form a vector under continuous transformations, such as rotations. Under the parity transformation, the matrix of derivatives is
$$
P=\frac{\partial x^{j}}{\partial x^{i^{\prime}}}=\operatorname{diag}(-1,-1,-1)
$$
and it is easy to see that $F_{0 i}$ behaves like a normal vector $V_{i}^{\prime}=-V_{i} .$ The second part of the exercise can be most easily solved by noting that $A_{k}=\frac{1}{2} \epsilon_{i j k} F^{i j} .$ As we know from the first part of the book, in this case, the components of $A_{k}$ do not change under parity, and hence they form an axial vector. Sometimes the notation $F_{\mu v}=(\mathbf{V}, \mathbf{A})$ is used.

\subsection{Relativistic dynamics of a free particle}
Consider the particle moving between the instants of time $t_{1}$ and $t_{2} .$ \bluep{In order to have Lorentz-covariant equations of motion, the action of the particle should be a constant scalar with respect to the Lorentz rotations (boosts), and also with respect to orthogonal transformations of space coordinates}. The simplest scalar expression that satisfies this condition has the form
\begin{equation}
S=-\alpha \int_{t_{1}}^{t_{2}} d s
\label{relativistic-action}
\end{equation}
where $\alpha$ is an unknown constant and one can find it by requiring that the nonrelativistic limit of the expression coincides with the action of a free particle in classical mechanics
\begin{equation}
S_{c l}=\int_{t_{1}}^{t_{2}} d t L_{c l}, \quad L_{c l}=T_{c l}=\frac{m v^{2}}{2}
\label{nonrelativistic-action}
\end{equation}
Let us first assume that $\beta\ll1$ and make a series expansion:
\begin{equation}
d s=c \sqrt{1-\beta^{2}} d t=c\left(1-\frac{\beta^{2}}{2}-\frac{\beta^{4}}{8}+\ldots\right) d t
\end{equation}
Replacing this expression into (\ref{relativistic-action}) and using $\beta=v/c$, we arrive at the nonrelativistic limit,
\begin{equation}
S_{n r}=\int_{t_{1}}^{t_{2}}\left(-\alpha c+\frac{\alpha v^{2}}{2 c}+\frac{\alpha v^{4}}{8 c^{3}} \cdot+\ldots\right) d t
\end{equation}
The first term in the parentheses is an irrelevant constant. The second term becomes exactly the desired expression (\ref{nonrelativistic-action}) if we assume that 
$$
\alpha=mc
$$
The later terms become small $\mathcal{O}\left(\beta^{4}\right)$ corrections. Since the value of $\alpha$ is now fixed, we can write down the final form of the action (\ref{relativistic-action}):
\begin{qt}
    \begin{equation}
S=-m c \int_{t_{1}}^{t_{2}} d s=\int_{t_{1}}^{t_{2}} d t L
\end{equation}
where
\begin{equation}
L=-m c^{2} \sqrt{1-\beta^{2}}
\label{lagrange-function}
\end{equation}
\end{qt}
The next steps will be to derive the expressions for canonical momenta and energy, and construct the Lagrange equations. In 4D formalism, remember that $d s=\sqrt{\eta_{\mu \nu} d x^{\mu} d x^{\nu}}$,$ds^{\prime}=\sqrt{\eta_{\mu \nu} d (x^{\mu}+\delta x^{\mu}) d (x^{\nu}+\delta x^{\nu})}$ and consider an arbitrary variation of four-coordinate $\delta x^{\mu}(t),$ such that
\begin{equation}
\delta x^{\mu}\left(t_{1}\right)=\delta x^{\mu}\left(t_{2}\right)=0
\end{equation}
Notice that $ds^{\prime 2}-ds^{2}=\delta s^2$,$\delta s=\frac{d(\delta s^2)}{2ds}$, the variation of the action is
\begin{equation}
\begin{aligned}
\delta S=-mc\int_{t1}^{t2}\delta s &=-m c \int_{t_{1}}^{t_{2}} \frac{1}{2 d s} \cdot\left(2 \eta_{\mu v} d x^{\mu} \delta d x^{v}\right)=-m c \int_{t_{1}}^{t_{2}} d s \eta_{\mu v} \frac{d x^{\mu}}{d s} \frac{\delta d x^{v}}{d s} \\
&=-\left.m c \eta_{\mu v} u^{\mu} \delta x^{v}\right|_{t_{1}} ^{t_{2}}+m c \int_{t_{1}}^{t_{2}} d s \omega_{\mu} \delta x^{\mu}
\end{aligned}
\end{equation}
In the last expression, we integrated by parts, used the standard relation $\delta d x^{v}=d \delta x^{v}$. The first term in the last expression vanishes. \bluep{As far as $\delta S$ should be zero for an arbitrary $\delta x^{\mu},$ the equations of motion for a free relativistic particle have the form}
\begin{qt}
    \begin{equation}
\omega^{\mu}=\frac{d u^{\mu}}{d s}=\frac{d^{2} x^{\mu}}{d s^{2}}=0
\end{equation}
\end{qt}
Now we can compare this result with the usual 3D treatment. Starting from the Lagrange function(\ref{lagrange-function}), one can take the derivative w.r.t. velocity,
\begin{equation}
p_{i}=\frac{\partial L}{\partial v^{i}}=\frac{m v^{i}}{\sqrt{1-\beta^{2}}} \Longrightarrow \mathbf{p}=\frac{m \mathbf{v}}{\sqrt{1-\beta^{2}}}
\label{relativisitic-momentum}
\end{equation}
which is the relativistic expression for the momentum of the particle. According to classical mechanics, the energy of the particle is
\begin{equation}
\varepsilon=\mathbf{p} \cdot \mathbf{v}-L=\frac{m c^{2}}{\sqrt{1-\beta^{2}}}
\label{relativistic-hamiltonian}
\end{equation}
\redp{The Lagrange equation $\frac{\partial L}{\partial q_{i}}(t, \boldsymbol{q}(t), \dot{\boldsymbol{q}}(t))-\frac{\mathrm{d}}{\mathrm{d} t} \frac{\partial L}{\partial \dot{q}_{i}}(t, \boldsymbol{q}(t), \dot{\boldsymbol{q}}(t))=0$ for the particle and energy conservation provide us with the relations}
\begin{equation}
\frac{d \mathbf{p}}{d t}=0, \quad \frac{d \varepsilon}{d t}=0
\end{equation}
Both expression (\ref{relativisitic-momentum}) and (\ref{relativistic-hamiltonian}) are very interesting and tell us a lot about relativity. First of all, the nonrelativistic $(\beta \ll 1)$ limits for momentum and energy are compatible with classical mechanics, plus small corrections,
\begin{equation}
\mathbf{p}=m \mathbf{v}+\frac{1}{2} m \beta^{2} \mathbf{v}+\mathcal{O}\left(\beta^{4}\right), \quad \varepsilon=m c^{2}+\frac{m v^{2}}{2}+\mathcal{O}\left(\beta^{4}\right)
\end{equation}
where
\begin{qt}
    $$\varepsilon_0=mc^2$$
\end{qt}
is called the rest energy of the particle. This formula is
indeed very important, because in relativistic quantum theory one can consistently describe how to transform the rest energy of the composite or even elementary particle into other forms of energy.

Finally, if we compare the Eq.(\ref{relativistic-u0}) and (\ref{relativistic-ui}) with (\ref{relativisitic-momentum}) and (\ref{relativistic-hamiltonian}), we arrive at the following identification for the four-momentum:
\begin{qt}
    \begin{equation}
p^{\mu}=m c u^{\mu}=\left(\frac{\varepsilon}{c}, \mathbf{p}\right)
\end{equation}
Using the identity $u^{\mu} u_{\mu}=1$ for the four-velocity, we also have the following \textbf{dispersion relation}:
\begin{equation}
p^{2}=p^{\mu} p^{v} \eta_{\mu v}=\frac{\varepsilon^{2}}{c^{2}}-\mathbf{p}^{2}=m^{2} c^{2}
\label{relativistic-dispersion}
\end{equation}
\end{qt}
An interesting feature of (\ref{relativistic-dispersion}) is that it applies not only to normal massive particles but also to massless particles and even to tachyons (assuming their masses
are complex).

\subsection{Charged Particle in Electromagnetic Field}
It proves useful to introduce a four-vector of electromagnetic potential $A^{\mu}(t, \mathbf{r})=(\varphi, \mathbf{A})$. And define:
$$
\mathbf{H}=\operatorname{rot} \mathbf{A}
$$
$$
\mathbf{E}=-\frac{1}{c} \frac{\partial \mathbf{A}}{\partial t}-\operatorname{grad} \varphi
$$
\bluep{$\mathbf{E}$ and $\mathbf{H}$ have six degrees of freedom (that means the number of independent functions), while $A^{\mu}(t, \mathbf{r})$ has only four of them. This reduction is one manifestation of the fact that the electric and magnetic fields are not independent, but form the unique electromagnetic field.}

The fact that scalar $\varphi$ and vector $A$ potentials form the four-vector $A^{\mu}$ is important, as it enables one to construct a covariant form of the interacting term for the charged particle in the form
\begin{equation}
S_{\mathrm{int}}=-\frac{e}{c} \int_{t_{1}}^{t_{2}} A_{\mu} d x^{\mu}
\end{equation}
Let us note that $A_{\mu}=(\varphi,-\mathbf{A})$ and hence $A_{\mu} d x^{\mu}=-\mathbf{A} \cdot d \mathbf{r}+\varphi c d t .$ Therefore,
in the $3 D$ notations the total action of the particle together with the interaction term can be written as follows:
\begin{qt}
    \begin{equation}
\begin{aligned}
&S=-\int_{s_{1}}^{s_{2}}\left\{m c+\frac{e}{c} A_{\mu} \frac{d x^{\mu}}{d s}\right\} d s=\int_{t_{1}}^{t_{2}} d t L\\
&L=-m c^{2} \sqrt{1-\beta^{2}}+\frac{e}{c} \mathbf{A} \cdot \mathbf{v}-e \varphi
\end{aligned}
\end{equation}
The canonically conjugated momentum is then
\begin{equation}
\mathbf{P}=\frac{\partial L}{\partial \mathbf{v}}=\frac{m \mathbf{v}}{\sqrt{1-\beta^{2}}}+\frac{e}{c} \mathbf{A}=\mathbf{p}+\frac{e}{c} \mathbf{A}
\end{equation}
the energy of the particle is
\begin{equation}
\varepsilon=\mathbf{P} \cdot \mathbf{v}-L=\frac{m c^{2}}{\sqrt{1-\beta^{2}}}+e \varphi
\end{equation}
\end{qt}
If we make the Legendre transformation and express $\varepsilon$ in terms of momentum as:
\begin{qt}
    \begin{equation}
H=\sqrt{m^{2} c^{4}+c^{2}\left(\mathbf{P}-\frac{e}{c} \mathbf{A}\right)^{2}}+e \varphi
\end{equation}
\end{qt}
Using Hamilton's equation $\mathbf{P} \rightarrow-\frac{\partial S}{\partial \mathbf{r}}$, we arrive at the Hamilton-Jacobi equation in the form
\begin{equation}
\left(\nabla S-\frac{e}{c} \mathbf{A}\right)^{2}-\frac{1}{c^{2}}\left(\frac{\partial S}{\partial t}+e \varphi\right)^{2}+m^{2} c^{2}=0
\end{equation}
Let us now use the variational principle to obtain the equations of motion. To this end consider an arbitrary variation of the coordinates of the particle $\delta x^{\mu}(t),$ which is frozen at the ends of the interval of integration, $\delta x^{\mu}\left(t_{1}\right)=\delta x^{\mu}\left(t_{2}\right)=0 .$ Then, as before,
\begin{equation}
\delta s=u_{\mu} \frac{d \delta x^{\mu}}{d s}
\end{equation}
hence the first variation of the action is:
\begin{equation}
\delta S=-\int_{s_{1}}^{s_{2}} d s\left\{m c u_{\mu} \frac{d \delta x^{\mu}}{d s}+\frac{e}{c} u^{\mu} \left(\partial_{\alpha} A_{\mu}\right) \delta x^{\alpha}+\frac{e}{c} A_{\mu} \frac{d \delta x^{\mu}}{d s} d s\right\}
\end{equation}
Integrating by parts and changing the names of indices, we arrive at
\begin{equation}
\delta S=\int_{s_{1}}^{s_{2}} d s\left\{m c \frac{d u_{\mu}}{d s}+\frac{e}{c} u^{\alpha}\left(\partial_{\alpha} A_{\mu}\right)-\frac{e}{c}\left(\partial_{\mu} A_{\alpha}\right) u^{\alpha}\right\} \delta x^{\mu}
\end{equation}
which provides the following equation of motion:
\begin{qt}
    \begin{equation}
m c \frac{d u_{\mu}}{d s}-\frac{e}{c} u^{\alpha} F_{\mu \alpha}=0
\end{equation}
In the last formula, we introduced a very important object
\begin{equation}
F_{\mu v}=\partial_{\mu} A_{v}-\partial_{v} A_{\mu}
\end{equation}
which is called as \textbf{tensor of electromagnetic field}.
\end{qt}
The explicit calculation shows that
\begin{equation}
F_{\mu v}=\left(\begin{array}{cccc}
{0} & {E_{x}} & {E_{y}} & {E_{z}} \\
{-E_{x}} & {0} & {H_{z}} & {-H_{y}} \\
{-E_{y}-H_{z}} & {0} & {H_{x}} \\
{-E_{z}} & {H_{y}} & {-H_{x}} & {0}
\end{array}\right)=(\mathrm{E}, \mathrm{H})
\label{electromagnetic-tensor}
\end{equation}
$H_{1,2,3}=-H^{1,2,3},$ and $E_{1,2,3}=-E^{1,2,3}$ are regarded as components of $4 D$ vector in Minkowski space. The first index $\mu$ in the $l . h . s .$ of (\ref{electromagnetic-tensor}) indicates the number of lines, while $v$ numbers the columns. It is clear from (\ref{electromagnetic-tensor}) that the tensor under discussion is antisymmetric $F_{\mu v}=-F_{v \mu} .$ For the space components $F_{i k},$ there is the following useful relation:
\begin{equation}
F_{i k}=\sum_{l=1,2,3} \varepsilon_{i k l} H_{l}, \quad \text { where } \quad H_{l}=H_{x}, H_{y}, H_{z}
\end{equation}
The example in the previous section shows that $\mathbf{H}$ is axial vector while $\mathbf{E}$ is the usual vector, which changes the sign of its components under parity transformation. By raising indices, we have
\begin{equation}
F^{\mu v}=\left(\begin{array}{cccc}
{0} & {-E_{x}} & {-E_{y}} & {-E_{z}} \\
{E_{x}} & {0} & {H_{z}} & {-H_{y}} \\
{E_{y}} & {-H_{z}} & {0} & {H_{x}} \\
{E_{z}} & {H_{y}} & {-H_{x}} & {0}
\end{array}\right)=(-\mathbf{E}, \mathbf{H})
\end{equation}
The dispersion relation between energy and momentum of a particle
interacting with electromagnetic potential becomes
\begin{equation}
\left(\frac{\varepsilon-e \varphi}{c}\right)^{2}=m^{2} c^{2}+\left(\mathbf{P}-\frac{e}{c} \mathbf{A}\right)^{2}
\end{equation}
Let's now introduce an absolutely antisymmetric Levi-Civita symbol $\varepsilon_{\alpha \beta \mu \nu}$ in Minkowski space. It is customary to define $\varepsilon^{\mathrm{0123}}=1$. With $\varepsilon_{\alpha \beta \mu \nu}$, we can prove that
\begin{qt}
    \begin{equation}
\tilde{F}^{\alpha \beta}=\frac{1}{2} \varepsilon^{\alpha \beta \mu \nu} F_{\mu \nu}
\end{equation}
where
\begin{equation}
\tilde{F}_{\alpha \beta}=\left(\begin{array}{cccc}
{0} & {H_{1}} & {H_{2}} & {H_{3}} \\
{-H_{1}} & {0} & {E_{3}} & {-E_{2}} \\
{-H_{2}} & {-E_{3}} & {0} & {E_{1}} \\
{-H_{3}} & {E_{2}} & {-E_{1}} & {0}
\end{array}\right)=(\mathbf{-H}, \mathbf{E})
\end{equation}
\end{qt}